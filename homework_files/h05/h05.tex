\documentclass{article}
%\usepackage[margin=0.5in]{geometry}
\usepackage{listings}

\title{Homework 05 - Interviews and Content Logs}
\author{Andrew Carter and Beryl Egerter}

\begin{document}
\maketitle
\section{Interview: ACBE3}
\subsection{Post Interview Reflection (Joint)}
\begin{itemize}
  \item Syntax error - immediately said was never on the line the error gives
  \item Should have asked people about their previous experiences such as grutoring, which could make a huge difference in looking at code
  \item Huge variety of skill levels we have been getting
  \item Much better at finding the problem without understanding everything else than previous participants
  \item Got both of the bugs in Ruby and continued, but stopped immediately after finding bug in python
  \item debugging problems, resorted to top down (except when given an error with line number)
  \item for the first two, really didn't need to write on paper to keep track of what was going on
  \item was thrown off by list syntax in python where code enclosed in brackets means different things in different places
\end{itemize}
\subsection{Interview Summary}
Interview length:  53:36m \\
No technical difficulties encountered.
We went through Questions 1.1, 1.2, 2.1, and 2.2, and 3.1 (3.1 cut off by time constraints). 
\subsection{Most Interesting Pieces}
\begin{itemize}
  \item Time:  \\
  Summary: \\
  Relations to Topic:  \\
  Possible Arguments: 
  \item Time:  \\
  Summary: \\
  Relations to Topic: \\
  Possible Arguments: 
\end{itemize}
\subsection{Content Log}
Content log is provided in separate text file labeled ACBE3.

\newpage
\section{Interview: ACBE4}
\subsection{Post Interview Reflection (Joint)}
\begin{itemize}
  \item Interesting that changed methods between 1.1 and 1.2
  \item also, differences between starting point in 1.1 and 1.2 vs. what large clues existed (main function v. not main function)
  \item If we had switched order of first and second would it have affected things?
  \item Very verbose
  \item Much better understanding of functions in 1.1 than in 1.2
  \item Assumptions that previous coders were intelligent
  \item Ideas of basic functions, or 'important' functions
\end{itemize}
\subsection{Interview Summary}
Interview length:  47:47 \\
Technical difficulties:
\begin{itemize}
  \item No color formatting on paper sheets. Didn't appear to be a serious problem, grayscale affect was visible to differentiate keywords so it wasn't a total loss of information.
  \item Time constraints meant we needed to hurry/not ask many follow up questions in Question 2.2 - also, participant stuck around briefly after camera was turned off, even though had given us a hard time constraint, which meant that part of the answer to the last question is not on tape.
\end{itemize}
We went through Questions 1.1, 1.2, 2.1, and 2.2.
\subsection{Most Interesting Pieces}
\begin{itemize}
  \item Time: 14:40 - 15:40 \\
  Summary:  Participant describes differences between Questions 1.1 and 1.2. Mentions that function calls branch more in 1.1. Also mentions that 1.1 has a main function that was a logical start of execution, but in 1.2, had to scan through code to find where statements first occur instead of function definitions. \\
  Relations to Topic: Gives a sense of what a programmer may first look for in code to find a place to start. \\
  Possible Arguments: While a programmer can just start from top going down, analyzing code, an efficient approach may be identifying the thread of execution by clues found in function names or syntax. 
  \item Time: 22:50 - 23:48 \\
  Summary: In Question 2.1, participant identifies is\_leap\_year? as a function to pay attention to. Then, by looking through all the code and seeing that the function is not called anywhere, participant claims that leap year support is not implemented. Participant did recognize that it was possible that actual code implemented without the function. \\
  Relations to Topic:  Shows how a programmer can pay attention to just top level constructs (function names and how functions interact) than the actual code computations. \\
  Possible Arguments: When debugging specific problems in unfamiliar code, keywords in function names and function calls are an efficient strategy.
\end{itemize}

\subsection{Content Log}
Content log is provided in separate text file labeled ACBE4.

\newpage
\section{Reflect on Expected Focus}
\section{New Focus?}
\end{document}

