\documentclass{article}
\title{Homework 03 - Project Revision}
\author{Andrew Carter and Beryl Egerter}

\begin{document}
\maketitle
\section{Scheduled Interviews}

\section{Description of Topic}

\section{Empirical Goals}

\section{Response to Participant Questions}

\section{Non-Linear Interview Questions}

\section{Interview Sequence and Questions}

\section{Handouts}
Attached below.
\newpage
{\Large
Question 1.1 \\
\line(1,0){300}
\begin{verbatim}
def func2(list, num):
        return func1(list, num, func4)

def func4(a, b):
        return a * b

def func1(list, num, f):
        acc = 0
        for i in list:
                acc += f(i, num)
        return acc

def main():
    print(func3([1,2,3,4]))

def func3(list)
    return func2(list, 4)

main()

\end{verbatim}
\newpage
Question 1.2 \\
\line(1,0){300}
\begin{verbatim}
def function50(i, L):
    return L[i+2]

def function37(L):
    return L[-1]+L

def function52(i):
    return function4() * i

def function1(j, k):
    return (j + k) * function52(1)

def function4():
    return 3

def function188(L):
    return function37(L)+function50(2, L)

def function0():
    return function188([1,2,3,4,5,6,7,8,9])[function1(0,1)]

function0()
\end{verbatim}

\newpage
Question 2.1, File 1 \\
\line(1,0){300}
\begin{verbatim}
#!/usr/bin/ruby

load "ourdate.rb"

d = OurDate.new(2011,1,4)
print "#{d.what_day}"
print "We started writing this file today.\n"
d.forward_time(365)
print "We are almost done now.\n"
print "#{d.what_day}"

\end{verbatim}
\newpage
Question 2.1, File 2 \\
\line(1,0){300}
\begin{verbatim}
#!/usr/bin/env ruby

$months31 = [1,3,5,7,8,10,12]
$months30 = [4,6,9,11]

class OurDate
    attr_accessor :year
    attr_accessor :month
    attr_accessor :day

    def initialize(year, month, day)
        @year = year
        @month = month
        @day = day
    end

    def is_equal?( d )
        puts @year == d.year and 
            @month == d.month and 
            @day = d.day
    end

    def is_leap_year?
        if @year % 400 == 0
            puts 1
        elsif @year % 100 == 0
            puts nil
        elsif @year % 4 == 0
            puts 1
        else
            puts nil
        end
    end

    def check_month
        if @month == 13
            @month = 1
            @year = @year + 1
        elsif @month == 0
            @month = 12
            @year = @year - 1
        end
    end

    def tomorrow
        @day = @day + 1
        if @day > 31
            for i in $months31
                if @month == i
                    @day = 1
                    @month = @month + 1
                    check_month
                end
            end
        elsif @day > 30
            for i in $months30
                if @month == i
                    @day = 1
                    @month = @month + 1
                    check_month
                end
            end
        elsif @day > 28 and @month == 2
            @day = 1
            @month = @month + 1
            check_month
        end
    end

    def yesterday
        @day = @day - 1
        if @day == 0
            @month = @month - 1
            check_month
            for i in $months31
                if @month == i
                    @day = 31
                end
            end
            for i in $months30
                if @month == i
                    @day = 30
                end
            end
            if @month == 2
                @day = 28
            end
        end
    end

    def forward_time(n)
        for i in 0..n
            tomorrow
        end
    end

    def reverse_time(n)
        for i in 0..n
            yesterday
        end
    end

    def what_day
        puts "Today is #{month}/#{day}, #{year}!"
    end
end

\end{verbatim}

}
\end{document}