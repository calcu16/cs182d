\documentclass{article}
\title{Homework 03 - Project Revision}
\author{Andrew Carter and Beryl Egerter}

\begin{document}
\maketitle
\section{Scheduled Interviews}
Our scheduled interviews:
\begin{enumerate}
  \item Interview with NW at 4:30pm on Friday, February 15
  \item
\end{enumerate}
\section{Description of Topic}

\section{Empirical Goals}

\section{Response to Participant Questions}
We realize that much of the code that we plan to ask participants to look at will be unfamiliar to them. Though we are interested in how they go about familiarizing themselves with the code, we do not intend to completely baffle them. Thus, we will answer any technical questions such as how the language work. We plan to answer any question that could be answered with a quick google search. We will not answer any questions about how the program is composed.



\section{Non-Linear Interview Questions}
Our questions are aimed to find a level that challenges the interviewee so that we can see the process more clearly. If an interviewee finds the Tier 1 Questions hard, we will not advance to Tier 2 Questions, and likewise for the Tier 2 Questions. We have prepared two questions each in Tier 1 and 2, and one question in Tier 3. Thus we can start with short, less complicated questions in Tier 1 and move up to Tier 3 which contains concepts the interviewee is unlikely to have seen if we believe the interviewee can handle it.

\section{Interview Sequence and Questions}
\begin{enumerate}
  \item Welcome participant.
  \item Read assent script to participant.
  \item Solicit and answer any questions the participant has.
  \item Have participant sign the consent form.
  \item Provide participant a copy of the consent form.
  \item Turn on the video camera.
  \item Have participant work on Question 1.1 \\
  \begin{enumerate}
    \item Introduce question:
    \item Hand participant the attached handout, Question 1.1
    \item 
  \end{enumerate}
  \item If participant was not confident on Question 1.1, have participant work on Question 1.2
  \item If participant was confident enough on Question 1.1 or 1.2, have participant work on Question 2.1
  \item If participant was not confident on Question 2.1, have participant work on Question 2.2
  \item If participant was confident enough on Question 2.1 or 2.2, have participant work on Question 3.1
  
\end{enumerate}

\section{Handouts}
Attached below.
\newpage
{\Large
Question 1.1 \\
\line(1,0){300}
\begin{verbatim}
def func2(list, num):
        return func1(list, num, func4)

def func4(a, b):
        return a * b

def func1(list, num, f):
        acc = 0
        for i in list:
                acc += f(i, num)
        return acc

def main():
    print(func3([1,2,3,4]))

def func3(list)
    return func2(list, 4)

main()

\end{verbatim}
\newpage
Question 1.2 \\
\line(1,0){300}
\begin{verbatim}
def function50(i, L):
    return L[i+2]

def function37(L):
    return L[-1]+L

def function52(i):
    return function4() * i

def function1(j, k):
    return (j + k) * function52(1)

def function4():
    return 3

def function188(L):
    return function37(L)+function50(2, L)

def function0():
    return function188([1,2,3,4,5,6,7,8,9])[function1(0,1)]

function0()
\end{verbatim}

\newpage
Question 2.1, File 1 \\
\line(1,0){300}
\begin{verbatim}
#!/usr/bin/ruby

load "ourdate.rb"

d = OurDate.new(2011,1,4)
print "#{d.what_day}"
print "We started writing this file today.\n"
d.forward_time(365)
print "We are almost done now.\n"
print "#{d.what_day}"

\end{verbatim}
\newpage
Question 2.1, File 2 \\
\line(1,0){300}
\begin{verbatim}
#!/usr/bin/env ruby

$months31 = [1,3,5,7,8,10,12]
$months30 = [4,6,9,11]

class OurDate
    attr_accessor :year
    attr_accessor :month
    attr_accessor :day

    def initialize(year, month, day)
        @year = year
        @month = month
        @day = day
    end

    def is_equal?( d )
        puts @year == d.year and 
            @month == d.month and 
            @day = d.day
    end

    def is_leap_year?
        if @year % 400 == 0
            puts 1
        elsif @year % 100 == 0
            puts nil
        elsif @year % 4 == 0
            puts 1
        else
            puts nil
        end
    end

    def check_month
        if @month == 13
            @month = 1
            @year = @year + 1
        elsif @month == 0
            @month = 12
            @year = @year - 1
        end
    end

    def tomorrow
        @day = @day + 1
        if @day > 31
            for i in $months31
                if @month == i
                    @day = 1
                    @month = @month + 1
                    check_month
                end
            end
        elsif @day > 30
            for i in $months30
                if @month == i
                    @day = 1
                    @month = @month + 1
                    check_month
                end
            end
        elsif @day > 28 and @month == 2
            @day = 1
            @month = @month + 1
            check_month
        end
    end

    def yesterday
        @day = @day - 1
        if @day == 0
            @month = @month - 1
            check_month
            for i in $months31
                if @month == i
                    @day = 31
                end
            end
            for i in $months30
                if @month == i
                    @day = 30
                end
            end
            if @month == 2
                @day = 28
            end
        end
    end

    def forward_time(n)
        for i in 0..n
            tomorrow
        end
    end

    def reverse_time(n)
        for i in 0..n
            yesterday
        end
    end

    def what_day
        puts "Today is #{month}/#{day}, #{year}!"
    end
end

\end{verbatim}

}
\end{document}