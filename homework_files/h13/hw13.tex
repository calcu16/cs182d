\documentclass{article}
\usepackage[margin=1in]{geometry}
\usepackage{listings}


\title{Homework 13 - Conclusion and Future Work}
\author{Andrew Carter and Beryl Egerter}

\setlength{\parindent}{0cm}
\begin{document}
\maketitle
\section{Conclusion}

% like introduction except with more about what we talked about
%what we found recap
From conducting interviews found that choosing the right strategy appeared to be useful for a student to solve the problems we gave. 
For the evaluation problem, the student chose execution order which seemed to help him understand how passing functions worked. 
For the debugging problem, the student chose to read from top down which allowed him to find a likely cause of the bug early and allowed him to skip a lot of complicated code.  \\

Each of these strategies was well suited for the problem. 
Different problems may have different optimal strategies. 
When learning computer science it will be beneficial for students to know multiple strategies in order to evaluate and debug their assignments. Likewise, industry programmers will be more effective at solving different types of problems if they know multiple strategies. \\

As a small qualitative study, our work has some shortcomings. We did not have a wide range of problems or interviewees, which elicited very few different strategies. For example, a problem similar to the debugging question we have analyzed but with the buggy function located past the complex code would confirm that the student's choice of reading from top down really helped him solve the problem more efficiently. Additionally, as a qualitative study with a limited amount of time devoted to data collection and analsysis, we were unable to create any significant conclusions. 

\section{Future Work}

% this is a study that could be done

% more qualitative research on determining viable strategies an then quantitative research on many things - how they are chosen, when do they work well, is there an optimal strategy,

For future work, we believe more qualitative research needs to be done to classify viable strategies for these two types of problems. 
Additionally, we would like to have a list of factors that may affect which strategy students choose for these problems. 
This could be done in a study where a wider variety of problems are given with changing factors such as length and complexity of code. 
\\

To follow the qualitative work, some quantitative research could occur. 
In particular, we would like to be able to quantify when strategies work well. 
We could learn if there is an optimal strategy for certain problems by requiring interviewees to use certain strategies even though they might choose another. 
Lastly, we could quantify how the factors in each problem affect which strategy is chosen by the student. 
\end{document}