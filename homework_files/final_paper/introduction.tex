\section{Introduction}

%motivation:
Programmers often have to look at code they are unfamiliar with for a variety of reasons, whether they are students or professionals.
In industry, programmers often work with a large code base, and fixing a bug or finding code that does something in particular may not be easy or entirely understandable.
In education, computer science students sometimes start programming assignments with a code base that they are expected to alter or add to, but do not necessarily have the background to understand all of it, and may be extremely confused if they cannot figure out where to change or add code.
Programmers with any amount of experience need effective strategies to understand existing code depending on the code and their objective. \\

%overview of design & open questions:
In our qualitative study, we asked two questions where each student needed to perform code evaluation and two where each student needed to perform code debugging.
In each of the questions, we embedded a concept we did not expect our participants to be familiar with or code that we thought would be hard to understand.
These included overly complicated functions, an unfamiliar programming language, or the functional programming concept of passing a function as an argument.
Including questions about both code evaluation and debugging was important to us since the research we have read has looked at code evaluation on its own or debugging on its own but not together with the same participant.
We wanted to see how strategies changed depending on the objective of the problem. \\

%what we found:
We found that a participant's choice of strategy could help them succeed in solving the problem by being able to avoid complicated code or understand unfamiliar concepts.
One participant in particular chose two different strategies for an evaluation problem and a debugging problem: following the order of execution and reading from the top of the code down.
The strategies helped him understand an unfamiliar concept in the evaluation problem and avoid complicated functions in the debugging problem. \\

%contribution & future work:
The primary contribution of this work is a qualitative analysis of how different strategies helped a student overcome unknown concepts both in code evaluation and in debugging.
As a small qualitative study, there are many questions that could be focused on next.
In particular, we are interested in other strategies that could be used for each problem, their effectiveness, and what leads students to choose which strategy.
We could also follow up on some factors that may have affected some of the decisions students made about strategy in this study such as length of code or access to resources such as books or search engines.\\

%\newpage