\section{Conclusions}
From conducting interviews, we found that choosing a good strategy appeared to be useful for a student to solve the problems we gave. 
For the evaluation problem, the student we analyzed chose the strategy of execution order, which seemed to help him understand how passing functions worked. 
For the debugging problem, the student chose to read from top down, which may helped him find a likely cause of the bug quickly and appeared to allow him to skip a lot of complicated code.  \\

Each of these strategies appeared to be well suited for the problem and the student appeared to choose both of them well, based on the limited overview of the problem he received before starting. 
From the two problems we analyzed, it appears that different problems may be better suited to different strategies and that the right strategy could be useful in understanding unfamiliar concepts. 
When learning computer science it will be beneficial for students to know multiple strategies in order to evaluate and debug their coding assignments. Likewise, industry programmers will be more effective at solving different types of problems if they know multiple strategies. \\

%As a small qualitative study, our work has some shortcomings. We did not have a wide range of problems or interviewees, which elicited very few different strategies. 
%While we did not create any significant conclusions as a result of this qualitative study, 
%For example, a problem similar to the debugging question we have analyzed but with the buggy function located past the complex code would confirm that the student's choice of reading from top down really helped him solve the problem more efficiently. Additionally, as a qualitative study with a limited amount of time devoted to data collection and analysis, we were unable to create any significant conclusions. 
%\newpage
