\section{Previous Research}

%intro paragraph goes here
A number of previously written papers are relevant to our work because they cover various studies about code debugging, evaluation, and the strategies used in either case. Our work builds on these papers by comparing debugging and evaluation strategies directly instead of treating them as separate topics. These papers also highlight some weaknesses of our study. \\

% andrew par 1
Mosemann and Wiedenbeck investigated code comprehension by novices with respect to navigation method \cite{mosemann2001}.
They gave students two different programs to evaluate and a navigation method to use: sequential, data flow, or control flow.
The students were evaluated by asking a series of yes/no questions about the program to assess their understanding of the code.
Mosemann and Wiedenbeck found no interaction between program and navigation method.
However, they suggested that with a more diverse set of programs and navigation methods there may be some interaction, but the study was not broad enough to discover them.\\

% andrew par 2
Fitzgerald, et al. investigated debugging methods among novices \cite{fitzgerald2008}.
They confirmed that students often used a tracing strategy on a variety of different debugging problems.
Their results show that the bugs in their study were relatively easy to fix, and that determining how programmers discover where a bug is located is more useful.
Our study found consistent results, and so we decided to focus on how the student discovers the location of a bug rather than how the student fixes it.
However, Fitzgerald, et al. note an increased reliance on online resources for debugging, something that we did not provide.
Some student strategies exhibited in their study that we could not possibly replicate in our study are using println statements and tracing using a debugger.\\

% beryl par 1
Rajlich and Wilde talk about concepts both as code structures, such as loops or passing a function, and domain knowledge, what the programmer wants the code to do \cite{1021348}.
From a series of case studies, they argued that experts do not understand the entirety of the code they are debugging, but simply map their domain knowledge onto code concepts to narrow down the code areas they must look at.
This mapping of domain knowledge to code appears to occur in interview we analyzed as the student connected functions within the code to actions taken in a connect four game.
These mappings allowed the student to focus only on a single function instead of understanding the entire program.\\

% beryl par 2
Through qualitative analysis of interviews, Mayrhauser and Vans noticed that program comprehension often involves multiple strategies of thinking about the code \cite{402076}.
In addition to explaining a number of strategies, how often they were used, and how they believe they work together, they performed a study of how often each of the strategies was used by a programmer looking for a section of code within a 40000 line program within two hours. They found that programmers often use multiple strategies to navigate the code, switching between them as needed.
While most of our interviews were far too short for our students to have the time or need to switch between different strategies, we did occasionally see strategies changing, and this paper provides information about what might occur if our problems and interviews had been longer. \\

% conclusion paragraph goes here
In previous research, some studies have looked at code evaluation or code debugging, and how strategies are used in evaluation or debugging \cite{mosemann2001}  \cite{fitzgerald2008} \cite{1021348} \cite{402076}.
%In previous research, some studies covered code evaluation and debugging strategies in a narrow range of programs, and concluded that the individual problem did not have an effect on strategy.
The researchers also discussed some weaknesses of our study. 
For instance, our students only had access to the code on paper, limiting more interactive debugging and web searches.
Additionally, our code samples were relatively short compared to what might be found in industry;
 longer code samples might have caused students to display different behavior.

\newpage
