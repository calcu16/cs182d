\documentclass{article}
\usepackage[margin=1in]{geometry}
\usepackage{listings}


\title{Homework 12 - Introduction and Methods}
\author{Andrew Carter and Beryl Egerter}

\setlength{\parindent}{0cm}
\begin{document}
\maketitle
\section{Abstract}
We performed a qualitative analysis on a study consisting of four interviews of college students with varying amounts of experience beyond basic programming. In the interviews, we asked the students to perform code comprehension or debugging on four or five code samples with varying levels of difficulty. Some code samples included programming concepts or syntax unfamiliar to the students. We are interested in the different strategies the students used for exploring unfamiliar code and have qualitatively analyzed how well the strategies worked on our code samples. Strategies we saw included students delving into the details of a function or obtaining a high level summary of a function from the name. Some students even changed strategies depending on the given task. We believe evaluating these strategies is important because the strategies can affect the success or failure of programmers trying to maintain, change, or interact with code written by others. These situations often occur in industry. Our study shows that programmers may change strategy based on problem type and complexity, of which future research may need to be careful about.

While many studies have looked at code comprehension strategies and many studies have looked at debugging methodology, we were interested in how a student's strategy changes between these tasks. We performed a study on code comprehension involving evaluation or debugging on different code samples. Evaluation is figuring out what a piece of code will output, while debugging is looking through a piece of code to identify the cause of a specific problem. For this poster we have qualitatively analyzed one student's response to an evaluation question and a debugging question. We found that while the student could have followed the execution of the program in both code samples, while evaluating, he followed the execution, but while debugging, he searched top down for where the problem could occur.  

\section{Introduction}

\section{Data Collection Methods}

\section{Data Analysis Methods}

\end{document}
