\documentclass{article}
\usepackage[margin=0.5in]{geometry}
\usepackage{listings}
\usepackage{color}
\usepackage[usenames,dvipsnames,svgnames,table]{xcolor}

\lstset{
	keywordstyle=\color{red},
	stringstyle=\color{blue},
	tabsize=2,
}

\title{Homework 09 - Transcript-Summary-Conclusion Table \& Appendix}
\author{Andrew Carter and Beryl Egerter}

\begin{document}
\maketitle

\section{Transcript-Summary-Conclusion Table}
\subsection{ACBE1, Q3}
\subsection{ACBE3, Q1}
\subsection{ACBE3, Q2}
\subsection{ACBE4, Q1}
\subsection{ACBE4, Q4}
\section{Appendix}
\subsection{Interview Question 1}
Code given to participants did not include line numbers. \\
\line(1,0){300}
\begin{lstlisting}[language=python]
0		def func2(list, num):
1			return func1(list, num, func4)
2
3		def func4(a, b):
4			return a * b
5
6		def func1(list, num, f):
7			acc = 0
8			for i in list:
9					acc += f(i, num)
10		return acc
11
12	def main():
13		print(func3([1,2,3,4]))
14
15	def func3(list):
16		return func2(list, 4)
17
18	main()

\end{lstlisting}
\newpage
\subsection{Interview Question 2}
Code given to participants did not include line numbers. \\
\line(1,0){300}
\begin{lstlisting}[language=python]
0		def function50(i, L):
1			return L[i+2]
2
3		def function37(L):
4			return [L[-1]]+L
5
6		def function52(i):
7			return function4() * i
8
9		def function1(j, k):
10		return (j + k) * function52(1)
11
12	def function4():
13		return 3
14
15	def function188(L):
16		return function37(L)+[function50(2, L)]
17
18	def function0():
19		return function188([1,2,3,4,5,6,7,8,9])[function1(0,1)]
20
21	x = function0()
22	print x

\end{lstlisting}

\newpage
\subsection{Interview Question 3}
Code given to participants did not include line numbers. \\
\line(1,0){300} \\
File 1: \\
\begin{lstlisting}[language=ruby]
0	#!/usr/bin/ruby
1	
2	load "ourdate.rb"
3	
4	d = OurDate.new(2011,1,4)
5	print "#{d.what_day}"
6	print "We started writing this file today.\n"
7	d.forward_time(365)
8	print "We are almost done now.\n"
9	print "#{d.what_day}"

\end{lstlisting} 
\vspace{1cm} File 2
\begin{lstlisting}[language=ruby]
0		#!/usr/bin/env ruby
1
2		$months31 = [1,3,5,7,8,10,12]
3		$months30 = [4,6,9,11]
4	
5		class OurDate
6			attr_accessor :year
7			attr_accessor :month
8			attr_accessor :day
9	
10		def initialize(year, month, day)
11			@year = year
12			@month = month
13			@day = day
14		end
15
16		def is_equal?( d )
17			puts @year = = d.year and 
18				@month = = d.month and 
19				@day = d.day
20		end
21
22		def is_leap_year?
23			if @year % 400 = = 0
24				puts true
25			elsif @year % 100 = = 0
26				puts false
27			elsif @year % 4 = = 0
28				puts true
29			else
30				puts false
31			end
32		end
33
34		def check_month
35			if @month = = 13
36				@month = 1
37				@year = @year + 1
38			elsif @month = = 0
39				@month = 12
40				@year = @year - 1
41			end
42		end
43
44		def tomorrow
45			@day = @day + 1
46			if @day > 31
47				for i in $months31
48					if @month = = i
49						@day = 1
50						@month = @month + 1
51						check_month
52					end
53				end
54			elsif @day > 30
55				for i in $months30
56					if @month = = i
57						@day = 1
58						@month = @month + 1
59						check_month
60					end
61				end
62			elsif @day > 28 and @month = = 2
63				@day = 1
64				@month = @month + 1
65				check_month
66			end
67		end
68
69		def yesterday
70			@day = @day - 1
71			if @day = = 0
72				@month = @month - 1
73				check_month
74				for i in $months31
75					if @month = = i
76						@day = 31
77					end
78				end
79				for i in $months30
80					if @month = = i
81						@day = 30
82					end
83				end
84				if @month = = 2
85					@day = 28
86				end
87			end
88		end
89
90		def forward_time(n)
91			for i in 0..n
92				tomorrow
93			end
94		end
95	
96		def reverse_time(n)
97			for i in 0..n
98				yesterday
99			end
100		end
101	
102		def what_day
103			puts "Today is #{month}/#{day}, #{year}!"
104		end
105	end

\end{lstlisting}
\newpage
\subsection{Interview Question 4}
Code given to participant did not contain line numbers. \\
\line(1,0){300}
\begin{lstlisting}[language=python]
0 	#!/bin/env python3
1
2 	class Board(object):
3 		def __init__(self, width=7, height=6):
4 			self.board = [[] for i in range(width)]
5 			self.width = 7
6 			self.height= 6
7 	
8 		def drop(self, player, column):
9 			if column < len(self.board):
10				self.board[column].append(player)
11				return True
12			return False
13	
14		def __str__(self):
15			result = ""
16			for r in reversed(range(self.height)):
17				result += "|"
18				for c in range(self.width):
19					if r < len(self.board[c]):
20						result += self.board[c][r]
21					else:
22						result += " "
23					result += "|"
24				result += "\n"
25			result += "-" * (2 * self.width + 1)
26			return result
27	
28		def full(self):
29			return all(len(col) >= self.height for col in self.board)
30	
31		def score(self, player):
32			for c in range(self.height):
33				for r in range(len(self.board[c])):
34					p = self.board[c][r]
35					for dc,dr in ((0,1),(1,0),(1,1),(1,-1)):
36						for i in range(1,4):
37							nc = c + i*dc
38							nr = c + i*dr
39							if nc < 0 or self.width <= nc:
40								break
41							if nr < 0 or len(self.board[nc]) <= nr:
42								break
43							if self.board[nc][nr] != p:
44								break
45						else:
46							return 1 if p = = player else -1
47			return 0
48
49	other = {'X' : 'O', 'O' : 'X'}
50	player = 'X'
51	board = Board()
52
53	while True:
54		try:
55			c = int(input("%s > " % player))
56		except TypeError:
57			continue
58		if not board.drop(player,c):
59			continue
60		print(board)
61		if board.score(player):
62			print("Player %s Wins!!!" % player)
63		elif board.full():
64			print("Tie")
65		else:
66			player = other[player]
67			continue
68		board = Board()
69		player = 'X'
70		print(board)
\end{lstlisting}

\end{document}