\documentclass{article}
\usepackage[margin=1in]{geometry}
\usepackage{listings}


\title{Homework 11 - Analysis Revision}
\author{Andrew Carter and Beryl Egerter}

\begin{document}
\maketitle
\section{Overview of Claims \& Introduction to Analysis}

\section{Play-by-Play}
\subsection{Q1}
\subsection{Q4}

\begin{tabular}{lp{13cm}}
1& Alright so we have a board class which is gonna presumably represent the connect 4 board [...]\\
\end{tabular} \\
The student relates the board class to the game described in the problem statement. \\ \\
\begin{tabular}{lp{13cm}}
2& it's going to create an array of arrays or a list of lists depending on what you call it in python [...] \\
3& i don't know if, if lists in python are dynamically allocated or not [...] \\
4& so i wonder what this is doing. I guess it would be creating - an array with at least six - er, seven secondary arrays in it [...]\\
\end{tabular}\\
The student is confused by arrays/lists in Python. \\ \\
\begin{tabular}{lp{13cm}}
5& so drop, i guess that is where you drop a piece in - if column less than self.width um self.board.append player \\
6&oh ok so its doing  [starts drawing on paper] like a - array of arrays essentially and then it just pops on a color [...] as they happen [...] \\
\end{tabular}\\
The student figures out what the board data structure looks like and how it changes. \\ \\
\begin{tabular}{lp{13cm}}
7&so this is where you would need to do the check [points at drop function with pen]\\
\end{tabular} \\
The student says the drop function is where the bug could be fixed.


\section{Interpretation}

The student is evaluating in execution order in Q1. The student is unfamiliar with the semantics of passing a function as an argument in Python. Initially, as demonstrated on line ?, he tries to call the function which led to confusion since there were no arguments. On line ?, he looks for implied arguments, further evidence that he is trying to call this function. However, by continuing in execution order to function 1, he is able to notice how the argument he had tried to call earlier is now being used as a function to apply, as stated in line ?.  \\

The student is debugging by reading from top down, stopping upon encountering a likely solution, in Q4. Upon reaching the creation of the board data member, the student's lack of experience with Python causes him to be confused about how the physical representation of a connect four board maps onto this data member. As he continues reading top down, the Python list functions used in the drop function help him determine which of the possibilities (previously enumerated on line ?) were true for the board data member. This helped him map the physical representation onto the data member. This knowledge helps him fix the bug which had been given to him in relation to the physical connect four board.  \\

The student used two different strategies in the debugging and evaluation question. These strategies helped him move past an unfamiliar concept each time. In other interviews, if the student did not move directly from function 2 to function 1 in Q1, it took much longer to connect function 4 passed as an argument as the applied function in function 1. In Q4, some other strategies we have seen have led the student to get stuck in other locations as they try to relate the code to the physical model. 





%The right method can help a student move past concepts they are not familiar with. While evaluating, a student used execution order which helped him understand a concept, and switching to top down let him ignore a large portion of the code and clear up some misunderstandings.


\section{Summary}

\end{document}
