\documentclass{article}
\usepackage[margin=1in]{geometry}
\usepackage{listings}


\title{Homework 11 - Analysis Revision}
\author{Andrew Carter and Beryl Egerter}

\begin{document}
\maketitle
\section{Overview of Claims \& Introduction to Analysis}

We will show that a student used different strategies to decide which order to look at functions between two problems.
For an evaluation problem, the student evaluated the code in the order it would have been executed.
For a debugging problem, the student started looking at the code at the top of the paper and continued down until reaching a likely candidate for the location of the bug.
For this student, each strategy appear to help him understand unfamiliar constructs, which then allows him to solve the problem.

\section{Play-by-Play}
\subsection{Q1}
\begin{tabular}{lcp{15cm}}
[1:55-1:58] & S & Alright so, its go a main, so that gonna start. \\

&& The student is going to start with the main function, which gets called at the bottom of the script.
In the intervening time he is going to continue in execution order. \\

[2:12-2:22] & S& 
	Function 2 returns function 1 with the same two arguments already passed to it,\\
&&	and function 4, the result of... \\

&& The student is evaluating function 2.
He notes that there is function 4, and he tries to determine what the resulting value should be. \\

[2:25] & S &
	Which doesn't have any implied arguments, thats interesting. Umm, \\
[2:35] & & thats, odd, \\

&& The student is looking for implied arguments to function 4. \\

[2:36-2:50] & S &
	lets see, so its calling functi-,ooh, it calling function 1.\\
&&	There we go. Uhh, yes, so its calling function with a list a number, the array and 4,\\
&&	and the function 4 as sort of a multiplier.\\
&&	A function to apply. \\

&& The student continues to function 1 and notices how the argument is being used.
He then deduces the correct semantics for the functin call in function 2. \\
\end{tabular}
\subsection{Q4}

\section{Interpretation}

The student is evaluating in execution order in Q1.
The student is unfamiliar with the semantics of passing a function as an argument in Python.
Initially, as demonstrated on line ?, he tries to call the function which led to confusion since there were no arguments. On line ?, he looks for implied arguments, further evidence that he is trying to call this function.
However, by continuing in execution order to function 1,he is able to notice how the argument he had tried to call earlier is now being used as a function to apply, as stated in line ?.  \\

The student is debugging by reading from top down, stopping upon encountering a likely solution, in Q4.
Upon reaching the creation of the board data member, the student's lack of experience with Python causes him to be confused about how the physical representation of a connect four board maps onto this data member.
As he continues reading top down, the Python list functions used in the drop function help him determine which of the possibilities (previously enumerated on line ?) were true for the board data member.
This helped him map the physical representation onto the data member.
This knowledge helps him fix the bug which had been given to him in relation to the physical connect four board.  \\

The student used two different strategies in the debugging and evaluation question.
These strategies helped him move past an unfamiliar concept each time. 
In other interviews, if the student did not move directly from function 2 to function 1 in Q1, it took much longer to connect function 4 passed as an argument as the applied function in function 1.
In Q4, some other strategies we have seen have led the student to get stuck in other locations as they try to relate the code to the physical model. 





%The right method can help a student move past concepts they are not familiar with. While evaluating, a student used execution order which helped him understand a concept, and switching to top down let him ignore a large portion of the code and clear up some misunderstandings.


\section{Summary}

\end{document}
